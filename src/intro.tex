Double beta decay ($\beta\beta$) is a very rare nuclear transition in which a nucleus with $Z$ protons decays into a nucleus with $Z + 2$ protons and the same mass number $A$. The decay can occur only if the initial nucleus is less bound than the final nucleus, and both more than the intermediate $Z + 1$ nucleus. There are 35 naturally-occurring isotopes that can undergo $\beta\beta$. Two decay modes are usually considered: (i) the standard two-neutrino mode ($\beta\beta2\nu$), consisting in two simultaneous beta decays, $(Z,A) \rightarrow (Z + 2,A) + 2 e^- +2 \nu$, which has been observed in several isotopes with typical half-lives in the range of $10^{18}$--$10^{21}$ years; and (ii) the neutrinoless mode (\bbonu), $(Z,A) \rightarrow (Z +2, A) + 2 e^-$, a hypothetical rare decay, which violates lepton-number conservation, and is, therefore, forbidden in the Standard Model.

The \bbonu\ process cannot occur unless neutrinos are Majorana particles. 
%
%Neutrinoless double beta (\bbonu) decay  is a hypothetical rare decay in which two neutrons undergo $\beta$ decay simultaneously and without the emission of neutrinos. %If realized in nature, this transition would be extremely rare, as we will amply discuss in this report. However, the importance of detecting it cannot be overstated. 
%
%The most constraining lower bounds on the \bbonu\ decay half-life for the five most important isotopes used for \bbonu searches are: 
%for \Ge{76}, \mbox{$T^{0\nu}_{1/2} > 1.8 \times 10^{26}$} years (at 90\% confidence level, from \cite{GERDA:2020xhi});  
%for \Xe{136} is \mbox{$T^{0\nu}_{1/2} > 2.3 \times 10^{26}$} years (at 90\% confidence level, from \cite{KamLAND-Zen:2022tow}); 
%for \Tl{130}, \mbox{$T^{0\nu}_{1/2} > 2.2 \times 10^{25}$} years (at 90\% confidence level, from \cite{CUORE:2021mvw}); 
%for \Mo{100}, \mbox{$T^{0\nu}_{1/2} > 18 \times 10^{24}$} years (at 90\% confidence level, from\cite{Augier:2022znx}); 
%and for for \Se{82}, \mbox{$T^{0\nu}_{1/2} > 3.2 \times 10^{23}$} years (at 90\% confidence level, from \cite{CUPID:2022puj}). 
%Essentially all these results improve, by at least one order of magnitude, the sensitivity achieved by previous experiments a decade ago. Notice that the most stringent limits, corresponding to \Ge{76} and \Xe{136} are pushing the lifetime of the \bbonu\ process beyond $10^{26}$ years.
%
%
%The importance of \bbonu\ searches goes beyond its intrinsic interest, as it is the only practical way to reveal experimentally that neutrinos are Majorana particles. 
If $\nu$ is a field describing a neutrino, stating that the neutrino is a Majorana particle is equivalent to saying that the charge-conjugated field --- that is, a field with all charges reversed --- also describes the same particle: $\nu=\nu^c$. If such Majorana condition is not fulfilled, we speak instead of Dirac neutrinos. 

The theoretical implications of experimentally establishing \bbonu\ would, therefore, be profound. In a broad sense, Majorana neutrinos would constitute a new form of matter, given that no Majorana fermions have been observed so far. Also, \bbonu\ observation would prove that total lepton number is not conserved in physical phenomena, a fact that could be linked to the cosmic asymmetry between matter and antimatter.  Finally, Majorana neutrinos would mean that a new physics scale must exist and is accessible in an indirect way through neutrino masses. 

In addition to theoretical prejudice in favor of Majorana neutrinos, there are other reasons to hope that experimental observation of \bbonu\ is at hand. Neutrinos are now known to be massive particles, thanks to neutrino oscillation experiments. If \bbonu\ is mediated by the standard light Majorana neutrino exchange mechanism, a non-zero neutrino mass would almost certainly translate into a non-zero \bbonu\ rate. While neutrino oscillation phenomenology is not enough \emph{per se} to provide a firm prediction for what such a rate should be, it does give us hope that a sufficiently fast one to be observable may be realized in Nature. 
%Furthermore, \bbonu\ may have been observed \emph{already}: there is an extremely intriguing, albeit controversial, claim for \bbonu\ observation in \GE\ that is awaiting unambiguous confirmation by future \bbonu\ experiments.

The importance of massive Majorana neutrinos 
%and the possibility that an experimental observation is at hand, 
has triggered a new generation of $\bb0\nu$~experiments spanning several isotopes and a rich selection of experimental techniques.
% including germanium calorimeters (GERDA, MAJORANA and the newcomer LEGEND), tellurium bolometers (CUORE), scintillating bolometers, which could be based on selenium or molybdenum isotopes (CUPID), time projection chambers (TPCs) based on liquid xenon (LXe) such as EXO-200 (and the projected nEXO apparatus), TPCs based on high pressure xenon gas (HPXe) pioneered by the NEXT experimental program, and large, self-shielding calorimeters, either based on xenon dissolved in liquid scintillator (KamLAND-Zen) or tellurium dissolved in liquid scintillator (SNO+). 
%
Five of them have published limits\footnote{Unless otherwise stated, all limits in this review are at 90\% CL.}
 on the lifetime of the \bbonu\ process ($T^{0\nu}_{1/2}$) that exceed $10^{25}$ years, increasing the sensitivity of previous  experiments by at least an order of magnitude. GERDA\footnote{\url{https://www.mpi-hd.mpg.de/gerda/}} and MAJORANA\footnote{\url{https://sanfordlab.org/experiment/majorana-demonstrator}} (target isotope \Ge{76}) find \mbox{$T^{0\nu}_{1/2} > 1.8 \times 10^{26}$} years \cite{GERDA:2020xhi} and \mbox{$T^{0\nu}_{1/2} > 8.3 \times 10^{25}$} years \cite{Majorana:2022udl} respectively; CUORE\footnote{\url{https://cuore.lngs.infn.it/en}} (target isotope \Tl{130}) obtains \mbox{$T^{0\nu}_{1/2} > 2.2 \times 10^{25}$} years \cite{CUORE:2021mvw}; EXO-200\footnote{\url{https://www-project.slac.stanford.edu/exo/}} and KamLAND-Zen\footnote{\url{https://www.ipmu.jp/en/research-activities/research-program/kamland}} (target isotope \Xe{136}) set limits of \mbox{$T^{0\nu}_{1/2} > 3.5 \times 10^{25}$} years \cite{EXO-200:2019rkq} and \mbox{$T^{0\nu}_{1/2} > 2.3 \times 10^{26}$} years \cite{KamLAND-Zen:2022tow} respectively. 
 
 In addition, the CUPID demonstrators have demonstrated the feasibility of scintillating bolometer detectors, which are currently envisioned as the ``successors'' of conventional bolometers (such as those used by CUORE) and have set limits to the \bbonu\ process for two relevant isotopes.
For \Mo{100}, CUPID-Mo sets a limit of \mbox{$T^{0\nu}_{1/2} > 1.8 \times 10^{24}$} years \cite{Augier:2022znx}
and for \Se{82}, CUPID-0 finds \mbox{$T^{0\nu}_{1/2} > 4.6 \times 10^{24}$} years \cite{CUPID:2022puj}. 
%Essentially all these results improve, by at least one order of magnitude, the sensitivity achieved by previous experiments a decade ago. Notice that the most stringent limits, corresponding to \Ge{76} and \Xe{136} are pushing the lifetime of the \bbonu\ process beyond $10^{26}$ years.

Also, during the last five years, the NEXT experiment\footnote{\url{https://next.ific.uv.es/next/}} has successfully demonstrated, through the operation of the NEXT-White demonstrator, the suitability of the technology based on electroluminescent high pressure xenon chambers (HPXeEL). In spite of the small mass of NEXT-White (3.5 kg of \Xe{136} in the fiducial region), the apparatus has carried out a search for \bbonu\ events, setting a limit of \mbox{$T^{0\nu}_{1/2} > 1.3 \times 10^{24}$} years  \cite{NEXT:2023daz}. The NEXT-100 apparatus, currently being commissioned at the Canfranc Underground Laboratory (LSC), deploys a much larger mass of \Xe{136} and will reach a sensitivity comparable to that of EXO-200.  

It is important to remark that, in spite of much increased sensitivity of the experiments discussed above, the resulting limits on the effective neutrino mass (see section \ref{sec:bb0nu} for a detailed discussion) are relatively modest. For example, the limit set by GERDA, translates into \mbox{$m_{\beta\beta} < 79-180$} meV. The push to explore smaller neutrino masses has resulted in a major experimental effort to build 
detectors which will deploy typically one order of magnitude larger masses than their previous incarnations, while at the same time reducing the background in the region of interest (ROI) by the same amount. In the process, GERDA and MAJORANA have joined into a single collaboration, LEGEND\footnote{\url{https://legend-exp.org}}, whose first phase, LEGEND-200, has already started data taking. CUORE has mutated into CUPID\footnote{\url{https://www.lngs.infn.it/en/pages/cupid-en}}, EXO-200 has inspired the nEXO proposal\footnote{\url{https://nexo.llnl.gov}}, which aims to scale up the LXe technology from about 200 kg to 5 tons, and the NEXT collaboration intends to follow up NEXT-100 with NEXT-HD, a ton-scale HPXeEL, and is actively researching the possibility to build a detector (dubbed NEXT-BOLD) capable of detecting the single Ba$^{2+}$ ion emitted in the decay as a mean to drastically suppress backgrounds. Also, the large scintillating calorimeters, KamLAND-Zen and SNO+\footnote{\url{https://snoplus.phy.queensu.ca}} are proposing upgrades which are both larger and able to control the background better. In addition, a host of R\&D proposals are being developed (see for example references in \cite{Dell_Oro_2016}). 
%Some of the experiments are already running or will run very soon. Some of them are still in their R\&D period. Some of them push to the limit the technique they use, in particular concerning the target mass. Others are easier to scale up. All of them claim to be sensitive to very light neutrino masses, by assuming that they can do one to three orders of magnitude better in background suppression and by significantly increasing their target mass, compared to previous experiments. 

In this report we review the state-of-the-art of this exciting and rapidly changing field. 
The organization of the paper is as follows. The introductory material is covered in sects.~\ref{sec:massivenu} and \ref{sec:bb0nu}. The key particle physics concepts involving massive Majorana neutrinos and neutrinoless double beta decay are laid out here. The current experimental knowledge on neutrino masses, lepton number violating processes in general, and \bbonu\ in particular, is also described in sects.~\ref{sec:massivenu} and \ref{sec:bb0nu}. Sections \ref{sec:nme}, \ref{sec:ingredients} and \ref{sec:experiments} cover more advanced topics. The theoretical aspects of the nuclear physics of \bbonu\ are discussed in sect.~\ref{sec:nme}. Sections \ref{sec:ingredients} and \ref{sec:experiments} deal with experimental aspects of \bbonu, and can be read without knowledge of sect.~\ref{sec:nme}. An attempt at a pedagogical discussion of experimental ingredients affecting \bbonu\ searches is made in sect.~\ref{sec:ingredients}. Section \ref{sec:experiments} adds a description of selected past, present and future experiments. 

