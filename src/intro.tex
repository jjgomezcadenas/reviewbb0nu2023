Neutrinoless double beta (\bbonu) decay  is a hypothetical nuclear transition in which two neutrons undergo $\beta$ decay simultaneously and without the emission of neutrinos. If realized in nature, this transition would be extremely rare: the most constraining lower bound on the \bbonu\ decay half-life, in \Ge{76} is \mbox{$T^{0\nu}_{1/2} > 1.8 \times 10^{26}$} years (at 90\% confidence level, from \cite{GERDA:2020xhi}), while in \Xe{136} is \mbox{$T^{0\nu}_{1/2} > 2.3 \times 10^{26}$} years (at 90\% confidence level, from \cite{KamLAND-Zen:2022tow}). Both results improve, by one order of magnitude, the sensitivity achieved by previous experiments a decade ago. 

The importance of \bbonu\ searches goes beyond its intrinsic interest, as it is the only practical way to reveal experimentally that neutrinos are Majorana particles. If $\nu$ is a field describing a neutrino, stating that the neutrino is a Majorana particle is equivalent to saying that the charge-conjugated field --- that is, a field with all charges reversed --- also describes the same particle: $\nu=\nu^c$. If such Majorana condition is not fulfilled, we speak instead of Dirac neutrinos. 

The theoretical implications of experimentally establishing \bbonu\ would be profound. In a broad sense, Majorana neutrinos would constitute a new form of matter, given that no Majorana fermions have been observed so far. Also, \bbonu\ observation would prove that total lepton number is not conserved in physical phenomena, a fact that could be linked to the cosmic asymmetry between matter and antimatter.  Finally, Majorana neutrinos would mean that a new physics scale must exist and is accessible in an indirect way through neutrino masses. 

In addition to theoretical prejudice in favor of Majorana neutrinos, there are other reasons to hope that experimental observation of \bbonu\ is at hand. Neutrinos are now known to be massive particles, thanks to neutrino oscillation experiments. If \bbonu\ is mediated by the standard light Majorana neutrino exchange mechanism, a non-zero neutrino mass would almost certainly translate into a non-zero \bbonu\ rate. While neutrino oscillation phenomenology is not enough \emph{per se} to provide a firm prediction for what such a rate should be, it does give us hope that a sufficiently fast one to be observable may be realized in Nature. 
%Furthermore, \bbonu\ may have been observed \emph{already}: there is an extremely intriguing, albeit controversial, claim for \bbonu\ observation in \GE\ that is awaiting unambiguous confirmation by future \bbonu\ experiments.

The profound theoretical implications of massive Majorana neutrinos, 
%and the possibility that an experimental observation is at hand, 
has triggered a new generation of $\bb0\nu$~experiments. At the time of writing this report, this new generation of experiments spans several isotopes, and a rich selection of experimental techniques, ranging from the well-established germanium calorimeters, to xenon time projection chambers. Four experiments (GERDA, CUORE, EXO-200 and KamLAND-Zen) have published their results, increasing the sensitivity to the \bbonu\ process by an order of magnitude with respect to previous experiments in three different isotopes (\Ge{76}, \Te{130} and \Xe{136}). Furthermore, the NEXT experiment has successfully demonstrated, through the operation of the NEXT-White apparatus suitability of a new technology based on electroluminescent high pressure xenon chambers and will be starting operations with a 100-kg detector (NEXT-100) in the next few months. Also, the SNO+ experiment is expected to start operations soon. 

At the same time, the push towards detectors which will deploy typically one order of magnitude larger masses than their previous incarnations, while at the same time reducing the background in the region of interest (ROI) by the same amount is already well under way. GERDA has evolved into LEGEND, whose first phase, LEGEND-200, is under construction and expected to take data soon. CUORE has mutated in CUPID, which will be based in bolometric scintillators, EXO-200 has spanned the nEXO proposal, which scales up the liquid xenon (LXe) technology from about 200 kg to 5 tons, the NEXT collaboration has proposed a ton-scale version of their high pressure electroluminescent xenon detector (HPXeEL), which could eventually be equipped with the means to detect the single Ba$^{2+}$ ion emitted in the decay, as a mean to drastically suppress backgrounds, and the large scintillating calorimeters, KamLAND-Zen and SNO+ are also proposing version of themselves which are both larger and able to control the background better. In addition, a host of R\&D proposals are being developed (see for example references in \cite{Dell_Oro_2016}). 
%Some of the experiments are already running or will run very soon. Some of them are still in their R\&D period. Some of them push to the limit the technique they use, in particular concerning the target mass. Others are easier to scale up. All of them claim to be sensitive to very light neutrino masses, by assuming that they can do one to three orders of magnitude better in background suppression and by significantly increasing their target mass, compared to previous experiments. 
In this report we review the state-of-the-art of this exciting and rapidly changing field. 

This review is organized as follows. The introductory material is covered in sects.~\ref{sec:massivenu} and \ref{sec:bb0nu}. The key particle physics concepts involving massive Majorana neutrinos and neutrinoless double beta decay are laid out here. The current experimental knowledge on neutrino masses, lepton number violating processes in general, and \bbonu\ in particular, is also described in sects.~\ref{sec:massivenu} and \ref{sec:bb0nu}. Sections \ref{sec:nme}, \ref{sec:ingredients} and \ref{sec:experiments} cover more advanced topics. The theoretical aspects of the nuclear physics of \bbonu\ are discussed in sect.~\ref{sec:nme}. Sections \ref{sec:ingredients} and \ref{sec:experiments} deal with experimental aspects of \bbonu, and can be read without knowledge of sect.~\ref{sec:nme}. An attempt at a pedagogical discussion of experimental ingredients affecting \bbonu\ searches is made in sect.~\ref{sec:ingredients}. Section \ref{sec:experiments} adds a description of selected new-generation experimental proposals, together with a comparison of their physics case. 

