%
Large liquid scintillator detectors such as SNO \cite{SNO:2002tuh}, KamLAND \cite{KamLAND:2002uet} or Borexino \cite{Borexino:2008gab} have a successful track record in low-background searches in neutrino physics. Loading them with large amounts of \bb\ isotopes represents a cost-effective way to search for neutrinoless double beta decay. Two collaborations are pursuing this approach: KamLAND-Zen and SNO+. Both experiments are reusing the existing detector infrastructure from previous reactor and solar neutrino experiments. 

The KamLAND-Zen experiment \cite{} is seaching for the \bbonu-decay of \Xe{136} using enriched xenon dissolved in liquid scintillator, a technique first proposed in 1994 \cite{Raghavan:1994qw}. The experiment reuses the neutrino KamLAND detector, located at the Kamioka Observatory, Japan. The KamLAND-Zen detector is composed of two concentric transparent balloons. The inner one, 3.8~m diameter and fabricated from 25 $\mu$m thick nylon film, contains 13 tonnes of liquid scintillator loaded with 745~kf of enriched xenon. The outer balloon, 13~m in diameter, contains 1~kilotonne of pure liquid scintillator, and serves as an active shield for external gamma background as well as a detector for internal radiation from the inner balloon. Buffer oil between the outer balloon and an 18~m diameter spherical stainless-steel containment tank shields the detector from external radiation. Scintillation light is recorded by 1325 17-in and 554 20-in photomultiplier tubes mounted on the stainless-steel tank, providing 34\% solid-angle coverage. The containment tank is surrounded by a 3.2-kt water-Cherenkov outer detector. KamLAND-Zen, which has been collecting physics data since late 2011, has published a measurement of the half-life of the \bbtnu\ decay of \Xe{136}, $2.38\pm0.02~(\mathrm{stat})\pm0.14~(\mathrm{syst})\times10^{21}$~years \cite{KamLANDZen:2012aa}, and a limit to the half-life of the \bbonu\ decay, $2.3\times10^{26}$~years (90\% CL)~\cite{KamLAND-Zen:2022tow,}. The energy resolution of the detector is 9.9\% FWHM at the $Q$ value of \Xe{136}. The achieved background rate in the region of interest is approximately $1.4\times10^{-4}$~\ckky, thanks to a tight selection cut in the fiducial volume and the identification of $^{214}$Bi events via Bi-Po tagging. KamLAND-Zen continues taking data, aiming at a sensitivity of $4.6\times10^{26}$~yr. The collaboration is planning a future upgrade to increase the photocathode coverage and improve light collection, which would allow them to reach energy resolutions close to 5\% (FWHM) at the \Qbb\ value. 

SNO+, the follow-up of the \emph{Sudbury Neutrino Observatory} (SNO), is a multipurpose liquid scintillator experiment housed in SNOLAB (Ontario, Canada). The detector reuses many of the components of its predecessor, replacing the heavy water by 780~tonnes of liquid scintillator in order to obtain a lower energy threshold. The detector consists of a 12~m diameter acrylic vessel surrounded by about 9500 8-in photomultiplier tubes that provide a 54\% effective photocathode coverage. The acrylic vessel is immersed in a bath of ultra pure water that fills the remaining extent of the underground cavern, attenuating the background from external media such as the PMTs and surrounding rock. The density of the liquid scintillator (0.86~g/cm$^{3}$) being lower than that of the surrounding water leads to a large buoyant force on the acrylic vessel. To keep it in place, a hold-down rope net has been installed over the detector and anchored to the cavity floor. The ultimate physics goal of the SNO+ experiment is to conduct a search for \bbonu\ in \Te{130}, which will be loaded into the liquid scintillator in the form of (non-enriched) telluric acid. A loading of 0.5\%, equivalent to 1.3~tons of \Te{130}, is planned for the first phase of the experiment. The energy resolution of the SNO+ detector is expected to be 10.5\% FWHM at the $Q$ value of \Te{130}. Consequently, the \bbtnu\ spectrum will be an important source of background. The expected levels of uranium and thorium in the liquid scintillator can also result in substantial activity near the \bbonu\ endpoint, mostly from the decays of $^{214}$Bi and $^{212}$Bi. Nevertheless, these can be, in principle, actively suppressed via Bi-Po $\alpha$ tagging. External backgrounds (not originating in the liquid scintillator) can be suppressed with a tight fiducial volume selection.
