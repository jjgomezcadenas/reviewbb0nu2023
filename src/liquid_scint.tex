%
Large liquid scintillator detectors such as SNO, KamLAND or Borexino have a successful track record in low-background searches in neutrino physics. Loading them with large amounts of \bb\ isotopes represents a cost-effective way to search for neutrinoless double beta decay. Two collaborations are pursuing this approach: KamLAND-Zen and SNO+. Both experiments are reusing the existing detector infrastructure from previous reactor and solar neutrino experiments. The approach, however, has a relatively poor energy resolution and limited particle identification capabilities. 

KamLAND-Zen is located in the Kamioka mine in Japan. The detector contains 13 tons of Xe-loaded liquid scintillator suspended in a transparent nylon-based inner ballon surrounded by 1 kton of liquid scintillator. The experiment currently holds the most sensitive limit on the effective Majorana neutrino mass. Using 380 kg of \Xe{136}, they set a lower limit on the \bbonu half-life of $1.07\times10^{26}$~yr, corresponding to $\mbb<0.061$--0.165~meV. KamLAND-Zen is currently taking data with an increased loading of 750~kg, aiming at a sensitivity of $4.6\times10^{26}$~yr.

SNO+ is repurposing the infrastructure used for the SNO experiment in SNOLAB. The detector will be filled with 800~tons of liquid scintillator loaded
with \Te{130}. Its high natural isotopic abundance avoids the enrichment process. In a first phase of the experiment, a loading of 0.5\% is expected, aiming at a sensitivity of $1.9\times10{26}$~yr after 5 years of data taking.

The KamLAND-Zen and SNO+ collaborations are planning future upgrades of their experiments to increase
their sensitivity. Both collaborations plan to increase the photocathode coverage and improve light collection, which would allow them to reach energy resolutions close to 5\% (FWHM) at the \Qbb\ value. 