The reliability of the sensitivity estimates given above critically depend on how realistic our choice of detector performance indicators is. In this section, we discuss how we have chosen the parameters reported in tab.~\ref{tab:parameters}. This discussion is mostly intended for the expert reader wishing to independently assess our choices, and to use his/her own judgment to modify them accordingly. Given that the largest uncertainty affecting the ultimate \mbb\ sensitivity of a proposal is almost always related the achievable background rates, we decide to quote a background rate range. For all other indicators, a single number rather than a range is used, see tab.~\ref{tab:parameters} 

The EXO-200 TPC is filled with about 175 kg of liquid xenon enriched to 80.6\% in the isotope \XE\ \cite{EXO-200:2011xzf}, corresponding to a \bb\ mass of about 141 \kgbb. For the \bbonu\ efficiency around \Qbb, we take the efficiency assumed by the EXO Collaboration for their \bbtnu\ analysis \cite{EXO-200:2011xzf}, corresponding to $\varepsilon = 0.34$ above the 720 keV analysis threshold. The inefficiencies are dominated by the fiducial volume cut, keeping 63 out of 175 kg of liquid xenon \cite{EXO-200:2011xzf} (0.36 efficiency). A 6.3\% inefficiency introduced by vetoing $\beta$-$\alpha$ coincidences \cite{EXO-200:2011xzf} has also been considered in our efficiency assumption. In \cite{EXO-200:2011xzf}, the collaboration measured an energy resolution of $\sigma_E/E=4.5\%$ at 2615 keV for the EXO-200 detector. This value was obtained using a 376 V/cm drift field and ionization signals only. An improvement of up to a factor of 2.5 could be achieved with higher (1--4 kV/cm) drift fields and combining ionization with scintillation information, see \cite{EXO-200:2003bso}. As a result, we assume a nearly-nominal, 4.2\% FWHM energy resolution at \Qbb, corresponding to about 100 keV. For our background rate lower limit, we consider the collaboration's goal of 20 radioactive background events per year in a $\pm 2\sigma$ interval around the \Qbb\ endpoint, for a $\sigma/E=1.6\%$ energy resolution at 2.5 MeV, a detector mass of 200 kg and a 80\% enrichment in \XE\ \cite{Hall:2010zzg}. From these numbers, we obtain a background rate of $c=0.78\times 10^{-3}$ \ckkbby. This nominal background rate prediction still remains to be updated based on real EXO-200 data. As a worst-case background rate scenario, we take the rate that has already been achieved: $4\times 10^{-3}$ \ckky, see \cite{EXO-200:2011xzf}. This background level was obtained without full lead shielding, radon exclusion tent, radon trap or full 3-dim reconstruction, and might therefore be improved in the future \cite{EXO-200:2011xzf}. This number corresponds to $c=5\times 10^{-3}$ \ckkbby\ per unit \bb\ mass.

GERDA-1 will use eight refurbished \GE\ diodes from the Heidelberg-Moscow and IGEX experiments, for a total active mass of 17.66 kg and 0.86 isotopic enrichment in \GE\ \cite{Knopfle:2012zz}, corresponding to a \bb\ mass of 15.2 \kgbb. As for the \bbonu\ efficiency, the actual value will ultimately depend on analysis details that are unknown at the moment, for example whether the collaboration will rely on pulse shape discrimination already in phase I to further reduce multi-site energy deposition events. In the absence of an updated number, we assume $\varepsilon =0.95$ as originally quoted by the collaboration in \cite{Abt:2004yk}. The FWHM energy resolution for GERDA-1 diodes was measured to be between 3.6 and 6.0 keV at the 2615 keV gamma ray line from \TL\ \cite{Cattadori:2012fy}. Taking the central value of this interval (4.8 keV) and extrapolating to the \GE\ \Qbb\ value (2.039 MeV), we estimate $\Delta E= 4.2\ \text{keV}$. The optimistic background rate scenario is assumed to be the collaboration's goal of 0.01 \ckky\ \cite{Abt:2004yk}, corresponding to 0.012 \ckkbby\ per unit \bb\ mass. GERDA started commissioning in mid-2010, and data obtained since then can be used to estimate a worst-case scenario for the achievable background rates. The best background rate measured with a string of 3 natural germanium detectors, refurbished from the Genius-TF experiment, is $0.06\pm 0.02$ \ckky\ \cite{Cattadori:2012fy}. Since mid-2011, the first enriched germanium detectors have been deployed on a second string arm, using the best detector configuration found so far. Preliminary data from the enriched germanium detectors indicate a background rate that is compatible with the one found with the natural germanium diodes \cite{Cattadori:2012fy}. As a consequence, we assume 0.06 \ckky\ as upper limit for the GERDA-1 background rate, translating into $c=0.07$ \ckkbby\ per unit \bb\ mass.

For the phase II of the experiment, the GERDA Collaboration purchased 37.5 kg of germanium with an isotopic abundance in \GE\ of 0.86. The material has already been purified into 35.4 kg of 6N germanium, corresponding to a \bb\ mass of 30.4 \kgbb. In order to reduce backgrounds, both sophisticated pulse-shape discrimination (PSD) techniques and additional instrumentation for the LAr veto are likely to be used by GERDA-2. We assume an overall \bbonu\ efficiency of $\varepsilon =0.84$, given by the product of the 0.86 efficiency for a PSD cut reported in \cite{Agostini:2010rd}, times the 0.973 efficiency for a LAr veto cut reported in \cite{heisel2011large}. Compared to phase I detectors, a significantly improved FWHM energy resolution of 2 keV at \Qbb\ has been measured for the BEGe detectors to be used by GERDA-2, see \cite{Agostini:2010ke}. The lower limit on the background rate is taken to be the collaboration's goal of 0.001 \ckky\ \cite{Abt:2004yk}, corresponding to $c=0.0012$ \ckkbby. Again, we use preliminary results from the GERDA commissioning runs to estimate an upper limit on the background rate. Commissioning data indicate that $\beta$ decays of $^{42}$K, that is in turn produced positively charged by the $^{42}$Ar decay within the LAr veto, can contribute very significantly to the \Qbb\ background rate. This $^{42}$K background can be most easily quantified by measuring the 1525 keV gamma ray line. On the one hand, the collaboration estimated via simulations that a background rate at \Qbb\ of up to $1.7\times 10^{-3}$ \ckky\ can be obtained by a homogeneous distribution of $^{42}$K around the detectors, for a 43.9 $\mu\text{Bq/kg}$ contamination in $^{42}$K \cite{lehnert2011analysis}. On the other hand, a 1525 keV line about 20 times more intense than expected was observed during the first commissioning run. A significant fraction of this enhancement has been understood as due to a inhomogeneous $^{42}$K distribution caused by the field lines drifting the positively-charged $^{42}$K ions to the detector surface. To prevent the $^{42}$K ions to reach the detector surfaces, the collaboration deployed a copper shield called the \emph{``mini-shroud''}. The additional shroud reduced the counts at the 1525 keV line and at \Qbb\ by a factor of 4-5, and a preliminary measurement of the $^{42}$Ar specific activity of about 160 $\mu\text{Bq/kg}$ was obtained for almost field-free runs \cite{Knopfle:2012zz}. This measured value, combined with the simulation result of $1.7\times 10^{-3}$ \ckky\ for a 43.9 $\mu\text{Bq/kg}$ contamination in $^{42}$K, motivates our upper limit background rate assumption of about 0.006 \ckky, or $c=0.007$ \ckkbby\ per unit \bb\ mass.

CUORE-0 will make use of a single CUORE-like tower with 39 kg of natural TeO$_2$ crystals and with an isotopic abundance of 0.34167 of \TE\ \cite{CUORE:2011boi}, corresponding to a \bb\ mass of 10.9 \kgbb. The overall signal efficiency has been estimated to be about $0.83$, as obtained for the big crystals of the CUORICINO experiment \cite{Andreotti:2010vj}. The main inefficiency source is the ``physical'' inefficiency due to beta particles escaping the detector and radiative processes. The expected FWHM energy resolution of the CUORE detectors is $\Delta E\simeq 5\ \text{keV}$ at the \bbonu\ transition energy \cite{CUORE:2011boi}. As lower limit on the background rate, we assume 0.05 \ckky\ from the 2.615 keV gamma ray multi-Compton events coming from the irreducible \THORIUM\ contamination of the CUORICINO cryostat, as done in \cite{CUORE:2011boi}. This rate corresponds to $c=0.18$ \ckkbby\ per unit \bb\ mass. As upper limit on the background rate, again we follow \cite{CUORE:2011boi} and assume 0.11 \ckky, translating into $c=0.39$ \ckkbby\ per unit \bb\ mass. This latter number includes an additional background contribution from scaling the CUORICINO background in the conservative case of a factor of 2 improvement in \URANIUM\ and \THORIUM\ contamination of the copper and crystal surfaces. Such surface contamination results in degraded alphas that may mimic the \bb\ signal. This factor of 2 improvement is largely motivated by the background rates measured in the Three Towers Test (TTT), allowing to estimate the surface contamination of the copper detector holders, responsible for a large ($50\pm 20\%$ \cite{CUORE:2011boi}) fraction of the CUORICINO backgrounds \cite{Alessandria:2011vj}. For such tests, crystals were dismounted from the CUORICINO detector, and repolished on the surfaces. Three different types of copper cleaning were tested. The copper treatment procedure selected by the collaboration proved to be able to reduce the copper surface contamination by at least a factor of 2 compared to CUORICINO \cite{CUORE:2011boi,Alessandria:2011vj}.  

The CUORE total active mass will be 741 kg for the 988 envisaged cubic detectors \cite{CUORE:2011boi}. As for CUORE-0, the natural TeO$_2$ crystals will have an isotopic abundance of 0.34167 of \TE\ \cite{CUORE:2011boi}, resulting in about 206 \kgbb\ of \TE. The overall signal efficiency and the FWHM energy resolution at \Qbb\ are assumed to be equal to the CUORE-0 figures given above, $\varepsilon =0.83$ and $\Delta E\simeq 5\ \text{keV}$, respectively. The main improvements in background rate reduction compared to CUORE-0 are assumed to come from a much cleaner cryostat and from the different detector geometry. Under these assumptions, the CUORE background will be dominated by copper and crystal surface contaminations. As lower limit on the background rate, the 0.01 \ckky\ goal of the collaboration is assumed \cite{CUORE:2011boi,CUORE:2003fpg}, corresponding to $c=0.036$ \ckkbby. As upper limit, an extrapolation of the currently available measurements \cite{Alessandria:2011vj} on copper and crystal surface contaminations to the CUORE geometry results in a 0.035 \ckky\ upper limit \cite{Gorla:2012gd}, resulting into $c=0.13$ \ckkbby\ per unit \bb\ mass.  

During the phase I of the experiment, KamLAND-Zen will use 389 kg of xenon \cite{kozlov2011status}, enriched to 0.917 \cite{koga2010kamland} in the isotope \XE, for a \XE\ total mass of 357 \kgbb. Concerning \bbonu\ efficiency, the collaboration will make use of a fiducial volume cut to suppress backgrounds that are external to the mini-balloon, for example due to \URANIUM/\THORIUM\ contaminations in the outer liquid scintillator or the environmental gamma ray background. Depending on the \URANIUM/\THORIUM\ contamination levels of the mini-balloon materials themselves (Nylon film, supporting ropes and pipes), the collaboration is also considering a tighter fiducial volume cut to suppress backgrounds originating from the mini-balloon. In the following, we assume that a fiducial volume definition placed 3~$\sigma_r$ inward with respect to the mini-balloon surface will be used, where $\sigma_r$ is the radial position reconstruction accuracy. Assuming $\sigma_r=\mathrm{12.5~cm}/\sqrt{E\mathrm{(MeV)}}$ \cite{kozlov2011status}, we estimate a corresponding fiducial volume efficiency of 0.61 for the 158 cm radius mini-balloon deployed. No other sources of inefficiency are considered in our estimate. Concerning energy reconstruction, the same performance measured in the previous phase of experiment is expected for KamLAND-Zen, given that the xenon-loaded liquid scintillator within the balloon has the same optical properties (light yield, transparency) as the KamLAND scintillator outside the balloon. An energy resolution of $\sigma_E/E=6.8\%/\sqrt{\textrm{E(MeV)}}$ is assumed \cite{kozlov2011status}, scaling to 250 keV FWHM energy resolution at \Qbb. For the background rate lower limit, we take the latest collaboration's expectations, obtained via simulations. In \cite{kozlov2011status}, 19.5 background events per year are expected. The three largest background sources are expected to be \bbtnu\ events entering the ROI, \BI\ events from the mini-balloon materials, and (to a lesser extent) $^{10}\textrm{C}$ events produced through cosmic muon spallation. Compared to previous estimates \cite{koga2010kamland}, a shorter \bbtnu\ half-life as measured in \cite{EXO-200:2011xzf} is assumed ($T_{1/2}\simeq 2\times 10^{21}$ years), together with a less radiopure mini-balloon (\URANIUM/\THORIUM\ concentrations of $3\times 10^{-12}$ g/g). This estimate results in a background rate of $c=0.22\times 10^{-3}$ \ckkbby. Factors that may potentially result in a higher-than-expected background rate are a non-perfect knowledge of the reconstructed energy spectrum of \bbtnu\ events spilling over the ROI, a higher background contribution from mini-balloon materials, and a worse tagging of \BI\ and $^{10}\textrm{C}$ backgrounds (estimated tagging efficiencies of 66\% and 90\%, respectively). It is, however, rather difficult to quantitatively estimate what a ``pessimistic'' background rate might be observed, at this stage. As in the SNO+ discussion below, we assume (to a large extent in a arbitrary fashion) a 8 times higher-than-expected background rate as upper limit. We note that more information to revisit the KamLAND-Zen background model should become available soon, given that KamLAND-Zen data-taking has started.

The MAJORANA demonstrator will contain 40 kg of germanium, of which at least 20 kg and up to 30 kg will be enriched to 86\% in \GE\ \cite{Majorana:2011vap}. We follow the collaboration's baseline and assume 20 kg of enriched germanium \cite{Wilkerson:2012ga}, corresponding to 17.2 \kgbb\ of \GE. A total \bbonu\ efficiency of 71\% is estimated, accounting for detector granularity, pulse-shape analysis (PSA), single-site time correlation (SSTC) and energy cuts. A 5\% loss due to edge effects and lost bremsstrahlung is also included in this number. We divide by the quoted 84\% efficiency of the energy cut alone to estimate a 85\% efficiency in tab.~\ref{tab:parameters}. For energy reconstruction, the 2 keV FWHM resolution measured at \Qbb\ in BEGe detectors \cite{Agostini:2010ke} is assumed, as for GERDA-2. For the background rate, we assume as lower limit the number quoted by the collaboration: 0.004 counts per year and kg in a 4 keV-wide ROI \cite{Majorana:2011vap}, corresponding to $c=1.2\times 10^{-3}$ \ckkbby. Main backgrounds responsible for this rate are expected to be prompt cosmogenics, and \TL/\BI\ contaminants in the cryostat and in the copper/lead shield. A non-negligible contribution from $^{68}$Ge contaminants in the enriched germanium crystals is also expected. The material purity specifications are extremely challenging, with contaminant goals at the $<0.1 (<0.3)\ \mu\mathrm{Bq/kg}$ level in \TL\ (\BI) for the electroformed copper to be used for detector mounts, cryostat and the inner copper shield, and about one order of magnitude worse for the commercial high-purity copper and lead to be used for the outer shield. Purities within one order of magnitude of such goals have already been demonstrated. We therefore assume a factor of 10 worse background rate than nominal as upper limit: $c=12\times 10^{-3}$ \ckkbby.

The SNO+ experiment will make use of 780 tonnes of liquid scintillator \cite{OKeeffe:2011dex}, with natural neodymium loading at the 0.1\% (w/w) \cite{Wright:2009csa}. Given the 5.6\% isotopic abundance of \ND, this concentration corresponds to 44 \kgbb\ of \bb\ emitter mass. The experiment is expected to have a fiducial volume cut rejecting about 50\% of the active volume \cite{Wright:2009csa}. No other sources of inefficiency are considered in our estimate. For energy reconstruction, we assume a light output of 400 NHit/MeV \cite{Wright:2009csa}, or 1347 NHit at \Qbb, where NHit is the number of PMT hits. This number is highly sensitive to the amount of neodymium concentration, and corresponds to 0.1\% (w/w) loading. Such a light output implies a FWHM resolution of about 220 keV at \Qbb. For the background rate, of the order of 100 background events per kton of liquid scintillator and per year are expected via simulations in a 200 keV energy window around \Qbb\ \cite{Wright:2009csa}. This background level translates into a rate of $c=0.009$ \ckkbby, which we take as lower limit. The three dominant backgrounds are expected to be $^8\textrm{B}$ solar neutrinos, \TL\ decays and \bbtnu\ events. The background level due to $^8\textrm{B}$ solar neutrinos is well-known. The \TL\ background assumes a level of \THORIUM\ impurities at the level measured by the BOREXINO experiment, $8.3\times 10^{-18}$ g/g. The collaboration expects energy reconstruction systematic effects, such as non-gaussian resolution tails, on the shape of the \bbtnu\ energy spectrum near the ROI to affect the sensitivity the most, see for example the study in \cite{Wright:2009csa}. Preliminary studies conducted by the collaboration indicate that a \mbb\ sensitivity within a factor of $5/3$ worse than the purely statistical sensitivity can be preserved including this systematic effect. In the large background approximation, which is valid for the SNO+ experiment, this systematic effect would therefore be equivalent to a background rate increase of up to a factor of $(5/3)^4\simeq 8$, resulting in a background rate upper limit of $c=0.07$ \ckkbby.

The NEXT detector will contain 99.14 kg of pressurized xenon gas in the TPC fiducial volume region, enriched to 0.90 in the isotope \XE\ \cite{NEXT:2011eyk}, corresponding to 89.2 \kgbb\ in \bb\ mass. A \bbonu\ efficiency of 0.25 has been estimated via simulations in \cite{NEXT:2011eyk}, accounting for inefficiencies in reconstructing the \bb\ track, and in imposing energy and topology cuts to suppress backgrounds. Given that the inefficiency introduced by the energy within ROI requirement is separately accounted for in our analysis, we assume an efficiency of $\varepsilon =0.25/0.76=0.33$ in tab.~\ref{tab:parameters}. Preliminary energy reconstruction measurements obtained with a kg-scale prototype yield a 4.6\% FWHM resolution at the 59.4 keV full energy peak for a $^{241}$Am calibration source \cite{NEXT:2011eyk}. This energy resolution extrapolates to 0.72\% FWHM resolution at \Qbb, or about 18 keV. As lower limit on the background rate, we take the collaboration's estimate of $0.2\times 10^{-3}$ \ckkbby, dominated by the \URANIUM/\THORIUM\ contamination of the titanium pressure vessel, conservatively assumed to be at the level of $200\ \mu\text{Bq/kg}$ for each isotope, see \cite{NEXT:2011eyk}. Such radiopurity assumptions are based upon the measured upper limits for the clean titanium used in the LUX cryostat \cite{Hall:2010zz}. As upper limit for the background rate, we take an independent (and more pessimistic) measurement of titanium radiopurity, at the level of $1.2\pm 0.4$ ($0.6\pm 0.3$) mBq/kg for \URANIUM\ (\THORIUM) using the Gator low-background counting facility at LNGS \cite{Baudis:2012bc,Baudis:2011am}. Contaminants in these amounts would translate into a background rate of about $10^{-3}$ \ckkbby\ at \Qbb. Also, the collaboration has started its own radiopurity R\&D campaign at LSC, with the goal of refining these assumptions in the near future.

For SuperNEMO, we assume a 7 kg mass in the isotope \SE, to be installed in the demonstrator module \cite{Shitov:2010nt}. A \bbonu\ efficiency of 0.28 has been estimated for a SuperNEMO module in a detailed study \cite{novella2009experimental}, accounting for acceptance, reconstruction efficiency and event selection efficiency. We assume this number in our estimates, which is in fact quite similar to the collaboration's goal of $\varepsilon =0.30$ \cite{Freshville:2011zz}. Calorimeter R\&D efforts achieved a $7.7\%/\sqrt{\textrm{E(MeV)}}$ FWHM energy resolution using a PVT scintillator directly coupled to a 8 inch, high QE, Hamamatsu PMT, see \cite{Freshville:2011zz}. This energy resolution measurement extrapolates to 4.4\% (or about 130 keV) FWHM at \Qbb. For the background rate, we take $c= 6\times 10^{-3}$ \ckkbby\ as worst-case scenario. This number comes from the preliminary result on the \bbonu\ search in \SE\ with the NEMO-3 detector \cite{simard2011results}: 14 events were observed (in agreement with background expectations) in the [2.6--3.2] MeV energy ROI, after 4.5 years of data-taking and 0.93 \kgbb\ of source foil. The backgrounds are dominated by \BI/\TL\ contamination of the foils (measured to be $530\pm 180\ \mu\mathrm{Bq/kg}$ and $340\pm 50\ \mu\mathrm{Bq/kg}$ in \SE, respectively, see \cite{NEMO:2009ewu}) and radon concentration in the tracking volume ($6.46\pm 0.02\ \mathrm{mBq/m}^3$ for the phase-2 of the experiment, see \cite{NEMO:2009ewu}). As background rate lower limit, we assume a factor of 10 improvement in both \TL/\BI\ radiopurity of the foils and in radon concentration in the tracker: $c=0.6\times 10^{-3}$ \ckkbby.
