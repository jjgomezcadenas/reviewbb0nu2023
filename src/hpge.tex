%
Germanium can be enriched in the \bb\ isotope \Ge{76}and transformed into high-purity germanium (HPGe) detectors, devices characterized by superb energy resolution and high efficiency. 

HPGe detectors have a long-standing record in neutrinoless double beta decay searches. In the 1990s, the two most sensitive experiments used germanium detectors: \textsc{Heidelberg-Moscow} (HM) ran in the Laboratori Nazionali del Gran Sasso (LNGS) and set a lower limit on the \bbonu\ half-life of \Ge{76} of $1.9 \times 10^{25}$~years (90\% CL) \cite{Klapdor-Kleingrothaus:2000eir}; the International Germanium Experiment (IGEX) operated in the Homestake Mine (USA), Canfranc (Spain) and the Baksan Neutrino Observatory (Russia), setting a slightly worse limit than HM, $T^{0\nu}_{1/2}(\Ge{76}) \geq 1.6 \times 10^{25}$~years (90\% CL) \cite{IGEX:2002bce}. A subset of the HM collaboration published a controversial claim of evidence for \bbonu\ decay \cite{Klapdor-Kleingrothaus:2001oba, Klapdor-Kleingrothaus:2006zcr}, which sparked at the time an intense debate in the community (e.g., ref.~\cite{Aalseth:2002dt}). 

HM and IGEX were succeeded by two new experiments, the GERmanium Detector Array (GERDA) \cite{GERDA:2020xhi} and the \textsc{Majorana Demonstrator} \cite{Majorana:2019nbd}, which employed novel types of HPGe devices with improved energy resolution and pulse-shape identification (i.e., detailed information about the topology of events through the time structure of the recorded charge signal). GERDA, located at LNGS, operated bare HPGe detectors in a high-purity instrumented liquid argon (LAr) cryostat, which provided not only the cooling for the HPGe devices, but also serves as active shielding and veto against external and internal background events. With a background index of $5.2\times10^{-4}$~\ckky\  and an energy resolution of $\sim3$~keV (FWHM) at $Q_{\bb}=2039$~keV \cite{GERDA:2020xhi}, GERDA is the first \bbonu-decay experiment to operate in virtually background-free conditions (i.e., less than one background count in the signal window). With a total published exposure of 127.2~kg~yr, the derived lower limit on the \bbonu\ half-life of \Ge{76} is $1.8\times10^{26}$~yr (90\% C.L.) \cite{GERDA:2020xhi}. The \textsc{Majorana Demonstrator} operated its HPGe detectors at the Sanford Underground Research Facility (SURF) in a high-purity shield built with electro-formed copper produced deep underground. After accumulating an exposure of 26.0~kg~yr, the experiment set a limit of $0.27\times10^{26}$~yr (90\% C.L.) \cite{Majorana:2019nbd}.

Building on the success of GERDA and the \textsc{Majorana Demonstrator}, the LEGEND collaboration aims to develop a staged \bbonu\ experimental program, incrementing the enriched germanium mass in 200--300~kg steps at a time. Its first phase, LEGEND-200, installed at LNGS reusing the GERDA cryostat, aims at a sensitivity to the \bbonu\ half-life of about $10^{27}$~yr, thanks to a projected background index of 0.6~counts/(FWHM~t~yr) and an exposure of 1 tonne~yr. The second phase, LEGEND-1000, aims at a sensitivity beyond $10^{28}$~yr by operating 1 tonne of enriched germanium detectors for an exposure of more than 10 t~yr at a background index of about 0.025~counts/(FWHM~t~yr). 
