\subsubsection*{CANDLES} This project \cite{Umehara:2010zz} proposes the use of CaF$_{2}$ scintillating crystals to search for \bbonu\ in \CA. The crystals would be immersed in liquid scintillator providing shielding and an active veto against external backgrounds. Among the \bb\ isotopes, \CA\ has the highest $Q$-value, 4.27 MeV. This places the signal well above the energy region of the natural radioactive processes. Unfortunately, the natural abundance of the isotope is only 0.187\% and enrichment seems complicated. Therefore, many tons of crystals are needed for a competitive new-generation experiment.  
%
\subsubsection*{COBRA} The COBRA experiment \cite{Zuber:2001vm, Zuber:2010zz} is exploring the potentials of Cadmium Zinc Telluride (CdZnTe) room-temperature semiconductor detectors for \bbonu\ searches. Out of the several \bb\ candidate isotopes in CdZnTe, COBRA is focusing on \TE, because of its natural abundance, and \CD, because of its high $Q$-value of 2.8 MeV. Activities are split in two main directions: (a) the identification of the main background components in a setup of 64 commercial 1-cm$^{3}$ CdZnTe diodes located at LNGS; and (b) the development of pixelized devices that would allow to reduce the background by particle identification.
%
\subsubsection*{DCBA} The Drift Chamber Beta-ray Analyzer \cite{Ishikawa:2011zza} is a magnetized tracker (drift chambers) that can reconstruct the trajectories of charged particles emitted from a \bb\ source foil. The momentum and kinetic energy are derived from the track curvature in the magnetic field. A prototype, DCBA-T2, has shown energy resolution of about 150 keV (FWHM) at 1 MeV, and the main source of background (\BI) has been identified. A new apparatus, DCBA-T3, with a more intense magnetic field is now under construction at KEK.
%
\subsubsection*{LUCIFER} The idea of LUCIFER \cite{Giuliani:2010zz, Ferroni:2011zz} is to join the bolometric technique proposed for the CUORE experiment with the bolometric light detection technique used in cryogenic dark matter experiments. Preliminary tests on several \bbonu\ detectors have clearly demonstrated the background rejection capabilities that arise from the simultaneous, independent, double readout (heat and scintillation). LUCIFER will consist of an array of ZnSe crystals operated at 20 mK. The proof of principle with about 10 kg of enriched Se is foreseen for 2014.
%
\subsubsection*{MOON} The MOON detector \cite{Ejiri:2010zz} is a stack of multi-layer modules, each one consisting of a scintillator plate for measuring energy and time, two thin detector layers for position and particle identification, and a thin \bb\ source film interleaved between them. At present, NaI(Tl) scintillators are considered as the candidates for the scintillator plates. Energy resolution around 3\% FWHM at 3 MeV has been achieved during the R\&D phase. For position-sensitive detectors, possible candidates are multi-wire proportional chambers (MWPCs) and Si-strip detectors. 
%
\subsubsection*{XMASS} XMASS \cite{Sekiya:2010bf, Takeda:2011zza} is a multi-purpose liquid xenon scintillator. Although optimized for dark matter searches, it will also investigate neutrinoless double beta decay and solar neutrinos. The detector, with about 800 kg of xenon, was installed in the Kamioka mine (Japan) in the fall of 2010. The excellent self-shielding capabilities of the liquid xenon will be used to define a virtually background-free inner volume. 
%%%