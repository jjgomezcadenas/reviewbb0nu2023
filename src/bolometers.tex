%
Bolometers for \bbonu-decay searches consist of a dielectric crystal, which contains the isotope of interest, coupled to a temperature sensor. When these crystals are cooled to very low temperatures ($<20$~mK for bolometers with masses in the 0.1--1~kg range), the energy deposited by interacting particles is measurable as a rise in temperature. The crystals, therefore, function as highly sensitive calorimeters. The bolometric technique can provide high detection efficiency (70\%--90\%) and high energy resolution (down to 0.15\% at the $Q$ value). In principle, most of the favourable \bb\ isotopes could be studied with this technique, but an isotope of great interest \Te{130}, as its natural abundance (34\%) is by far the highest among all the \bbonu-decay candidates. 

The experiment CUORE, located at LNGS and currently in data taking, consists of 988 TeO$_2$ bolometers (containing tellurium with a natural isotopic composition) with a mass of about 750 g each, corresponding to about 200 kg of \Te{130}. CUORE has set a limit of $1.5\times10^{25}$~years on the half-life of \Te{130}. This limit will be improved by about a factor 2 at the conclusion of the 4 years CUORE physics program. The background in the ROI of CUORE, corresponding to about 50~events/year, is dominated by energy-degraded $\alpha$ particles generated by surface contamination. 

Scintillating bolometers could bring an additional value to the calorimetric technology. In these devices the crystal containing the isotope of interest is a scintillator, and a second auxiliary bolometer to register the emitted scintillation light is operated close to it. The ensemble of the crystal containing the \bbonu-candidate and its light detector is referred to as detector module. The simultaneous detection of heat and scintillation light allows one to distinguish $\alpha$ particles from electrons or gamma rays thanks to the different light yield and signal shape, eliminating the dominant background source observed in CUORE. 
Candidates that fit the high-Q-value requirement and can be embedded in scintillating crystals are 82Se (Q-value = 2998 keV, compound ZnSe), 100Mo (Q-value = 3034 keV, compound Li2MoO4) and 116Cd (Q-value = 2813 keV, compound CdWO4). 

CUPID (CUORE Upgrade with Particle IDentification) is a proposed next-generation \bbonu-decay experiment based on scintillating bolometers to be installed in the cryogenic infrastructure currently hosting CUORE at LNGS. The bolometer crystals will be grown from Li100MoO4 enriched to 95\% in Mo-100. The combination of the results achieved during the R\&D phase allows to predict a background index of the order of $10^{-4}$ \ckky\ in the ROI of Mo-100. The CUPID discovery sensitivity on \mbb\ is 12--20 meV in 10 y live time.
Upgrades are possible beyond the currently proposed version of CUPID. The present background model indicates the directions to be taken in order to reduce the background index by at least further order of magnitude, bringing it to the level of $10^{-5}$ \ckky or less. Another possible scenario is an ultimate bolometric detector, CUPID-1T, consisting of 1.8 tons of Li2MoO4, or 1000 kg of 100Mo. Such detector could be accommodated in a new cryostat approximately 4 times larger than CUORE. For optimal sensitivity, the background should be further reduced to the level of $5\times10^{-6}$ \ckky. The 3$\sigma$ half-life discovery sensitivity of these two future searches would be $2\times10^{27}$~yr and $8\times10^{27}$~yr respectively in 10~yr of data taking.