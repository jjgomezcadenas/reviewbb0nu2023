%%%%%
For almost half a century the only evidence of the existence of double beta decay came from geochemical methods consisting in measuring the concentrations of the stable daughter isotopes $(Z+2,A)$, produced over geologic times ($\sim 10^9$ years). An excess of the daughter isotope over its natural concentration is interpreted as evidence for \bb\ decay (either \bbtnu\ or \bbonu, since the method cannot distinguish between them).

The first direct measurement of \bbtnu, in \SE, did not happen until 1987 \cite{Elliott:1987kp}. It was done using a fairly large ($\sim1$ m$^{3}$) time projection chamber, the well-known Irvine TPC. The source, 14 g of 97\% enriched \SE, was deposited on a thin Mylar foil forming the central electrode of the chamber. The trajectories of the electrons emitted from the source foil were recorded by the TPC and analyzed to infer their energy and kinematic characteristics. Since this initial detection, the two-neutrino mode has been directly observed for 8 isotopes in several experiments (see table \ref{tab:bb2nu_exp} and ref.~\cite{Barabash:2010ie} for further details).

The most restricting limits to date in the search for \bbonu\ were obtained with germanium detectors. The Heidelberg-Moscow (HM) experiment \cite{Klapdor-Kleingrothaus:2000eir} searched for the \bbonu\ decay of \GE\ using five high-purity Ge semiconductor detectors enriched to 86\% in \GE. The experiment ran in the Laboratori Nazionali del Gran Sasso (LNGS), Italy, from 1990 to 2003, totaling an exposure of 71.7 kg$\cdot$year. The background rate reached by the experiment in the \Qbb\ region was ($0.19\pm 0.01$) \ckky, or, in units of \bb\ emitter mass, $0.22\pm 0.01$ \ckkbby. Pulse shape discrimination (PSD) was used in a subset of the data (35.5 kg$\cdot$year) to separate single-site events, like \bbonu\ decays, from multi-site events, like $\gamma$ interactions, resulting in a background rate of ($0.06\pm 0.01$) \ckky, or ($0.07\pm 0.01$) \ckkbby. A lower limit on the \bbonu\ half-life of $T^{0\nu}_{1/2}(\GE) \geq 1.9 \times 10^{25}$ years (90\% CL) was obtained \cite{Klapdor-Kleingrothaus:2000eir}.

A subset of the collaboration re-analyzed the data claiming evidence for \GE\ \bbonu\ decay \cite{Klapdor-Kleingrothaus:2001oba}. The latest publication by this group reports a $6\sigma$ evidence for \bbonu\ and a half-life measurement of $T_{1/2}^{0\nu}=(2.23^{+0.44}_{-0.31})\times 10^{25}$ years \cite{Klapdor-Kleingrothaus:2006zcr}, corresponding to $\mbb\ = (0.30^{+0.02}_{-0.03})\ \text{eV}$ according to the central value of the PMR nuclear matrix element for \GE\ given in sect.~\ref{subsec:nme_pmr}. This claim sparked an intense debate in the community, and at the moment no consensus exists about its validity (see, for example, ref.~\cite{Aalseth:2002dt}).

The International Germanium Experiment (IGEX) \cite{IGEX:2002bce} also searched for \bbonu\ using enriched germanium crystals. It ran in the Homestake gold mine (USA), the Canfranc Underground Laboratory (Spain) and the Baksan Neutrino Observatory (Russia) from 1991 to 2000, accumulating a total exposure of 8.87 kg$\cdot$year. It reached a sensitivity similar to that of Heidelberg-Moscow, but not enough to disprove the claim. The lowest background rate reached by the IGEX experiment was 0.26 (0.10) \ckky\ without (with) pulse shape discrimination for a 8.87 (4.65) kg$\cdot$year total exposure \cite{Gonzalez:2003pr}, corresponding to 0.30 (0.12) \ckkbby\ per unit \bb\ emitter mass.

The Cuoricino experiment, an array of 62 TeO$_{2}$ bolometric crystals, ran for five years in Gran Sasso searching for \bbonu\ in \TE. It reached a sensitivity to \mbb\ comparable to that of the HM experiment, but it cannot disprove the claim due to the uncertainties in the nuclear matrix elements. The average background rate for the 5$\times$5$\times$5 cm$^3$ Cuoricino crystals, computed in a 60 keV wide region centered around \Qbb , was $0.161\pm 0.006$ \ckky\ \cite{CUORE:2011boi}, corresponding to $0.58\pm 0.02$ \ckkbby\ per unit \bb\ emitter mass. The average FWHM energy resolution in all crystals was $6.3\pm 2.5$ keV at 2615 keV \cite{CUORE:2011boi}.

The lowest levels of background so far were achieved by the NEMO3 experiment \cite{NEMO:2009ewu}: a few times $10^{-3}$ \ckky . This detector represents the state of the art of separate-source \bb\ experiments. Reconstruction of the electron tracks emerging from the source provided a powerful signature to discriminate signal from background. The NEMO3 experiment ran from 2003 to 2010 at the Modane Underground Laboratory (LSM), in France. The detector, of cylindrical shape, had 20 segments of thin source planes, with a total area of 20 m$^{2}$, supporting about 10 kg of source material. The sources were within a drift chamber, for tracking, surrounded by plastic scintillator blocks, for calorimetry. A solenoid generated  a magnetic field of 25 Gauss which allowed the measurement of the tracks electric charge sign. The detector was shielded against external gammas by 18 cm of low-background iron. Fast neutrons from the laboratory environment were suppressed by an external shield of water, and by wood and polyethylene plates. The air in the experimental area was constantly flushed, and processed through a radon-free purification system embedding the detector volume. In addition to searching for \bbonu , NEMO3 very successfully served as a ``\bbtnu\ factory'', providing precise \bbtnu\ half-life measurements for seven \bb\ isotopes, see table \ref{tab:bb2nu_exp}. Apart from representing the ultimate background for \bbonu\ searches , an accurate measurement of \bbtnu\ in several nuclides is also important as input to NME calculations. 